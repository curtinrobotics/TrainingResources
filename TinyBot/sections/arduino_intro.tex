\documentclass[../TinyBot.tex]{subfiles}
\begin{document}

\section{Intro to Programming}

This section will briefly explain what programming is, if you have programmed before, or feel confident in your programming knowledge, feel free to skip to the \href{sec:introarduino}{next section}.

If you have 0 programming experience, and are very confused by this section you may find it worthwhile to google programming guides and tutorials to really help you understand how to code and how code works. There are some links you may find useful at the end of this section. \\


Programming is how we tell computers what we want them to do. We can program a computer to blink a light, play a noise every time something comes too close, or drive a robot around. The set of instructions we write is called code, and so programming is also called coding. 
Like with spoken languages, there are many different programming languages. Popular languages include Java, C++, and Python. This guide will be introducing C++. \\

Each coding language has a specific structure that must be followed, called syntax. For the computer to understand your code, it must conform to the expected syntax exactly. Syntax errors occur when there is a comma somewhere there shouldn't be, a word that is capitalised when it shouldn't be, there's an extra bracket, etc.

The computer will tell you when there's a syntax error, and will often tell you what line the error is on. Sometimes this line number is a bit off, the syntax error might be a few lines above the specified line. When you first start coding, noticing where there is a syntax error is quite difficult; however as with many things, as you get more used to programming you get better at noticing where the syntax error is. \\


A crucial part of programming is saving data, you want to be able to save and store some data that you can use later. This data is called a variable, and because computers require specific syntax, you need to tell the computer exactly what type of data the computer is being told to remember.\\

Here we tell the computer to remember an \lstinline[]!int!eger variable called \lstinline[]!number1! which has the value 4.
\begin{lstlisting}
int number1 = 4;
\end{lstlisting}

The keyword \lstinline[]!int! is really important, it tells the computer that the variable \lstinline[]!number1! is an integer (a whole number, negative or positive). \\

Some other basic keywords (known as datatypes) that can be used in place of \lstinline[]!int! are:
\begin{itemize}[label={$\triangleright$}]
  \item \lstinline[]!float! - a decimal number such as 0.5, 3.14159, etc. 
  \item \lstinline[]!char! - a character, such as 'a' or '2' - note the single quotation marks which are important when declaring (creating) a character variable
  \item \lstinline[]!string! - a word or lots of words e.g. "hello world" or "this is a string"; double quotation marks are essential for strings. 
\end{itemize}

These datatypes are a common concept across many languages, though some languages don't require you to explicitly state what datatype a variable is. \\

Some more C++ resources:
\begin{itemize}[label={$\triangleright$}]
  \item \href{https://www.w3schools.com/cpp/default.asp}{https://www.w3schools.com/cpp/default.asp}
  \item \href{https://www.learncpp.com/}{https://www.learncpp.com/}
\end{itemize}

\section{Intro To Arduinos} \label{sec:introarduino}

To program an Arduino, you will need a USB cable, and a laptop/computer with the \href{https://www.arduino.cc/en/software}{Arduino IDE} installed, IDE stands for Integrated Development Environment. 



\begin{figure}[h]
    \centering
    \includegraphics[width=0.6\textwidth]{arduino_ide.jpg}
    \caption{The Arduino IDE}
\end{figure}

The two round buttons in the top right, the tick and the arrow, are the verify and upload buttons. Verify checks your code, making sure that the syntax (the structure of the code, think of it like a grammar checker) of your code is correct. Upload sends the code you've written to the Arduino board.  \\

However, before you can upload your code there's some setup you need to do. First of all, you need to select the board by going into the Tools menu, as shown in the image below. 

\begin{center}
  \includegraphics[width=0.6\textwidth]{arduino_ide_select_board.jpg}
  \captionof{figure}{Selecting Board}
\end{center}

Next, you need to select the USB port that the Arduino is connected to. This is also done through the Tools menu. The port should appear as COM followed by a number. This number will change depending on which USB port the Arduino is plugged into. 

\begin{center}
  \includegraphics[width=0.6\textwidth]{arduino_ide_select_port.jpg}
  \captionof{figure}{Selecting Port}
\end{center}

\begin{warningbox}
  If you cannot select the port because it is greyed out try plugging the Arduino into a different USB port. If this doesn't work, try unplugging and plugging in the Arduino end of the USB cable. \\

  If you still cannot select a port, it may be an issue with the cable or with the Arduino. Use a different cable, or borrow someone else's Arduino to test if they are the issue. \\

  If you still cannot select the port, even after trying different USB ports, cables, and Arduinos; the issue is likely with your computer, reinstalling the Arduino IDE may help. 
  
\end{warningbox}


\bigskip

Arduino's are programmed in the programming language \lstinline[]!C++!; though there are a few differences. The below code section details a few features of coding.


\begin{lstlisting}
// this is a comment, comments are not read by the computer and can be anything you want

// variables declared not in a function will be accessible
// in all functions
int global_var = 0;

void setup {
  // everything in this function will run once

  // this code will run when the board is powered on,
  // or when the reset button is pressed
}

void loop {
  // everything in this function will run repeatedly
}

\end{lstlisting}
\bigskip

% \pagebreak
A useful feature of the Arduino IDE is all the example code which is provided.
\begin{center}
    \includegraphics[width=0.7\textwidth]{arduino_ide_blink_example.jpg}
    \label{fig:ide-blink}
    \captionof{figure}{Arduino IDE Example Code}
\end{center}

The simplest Arduino example is the Blink code, which turns on and off an onboard LED.

\begin{lstlisting}
void setup() {
  // initialize digital pin LED_BUILTIN as an output.
  pinMode(LED_BUILTIN, OUTPUT);
}

// the loop function runs over and over again forever
void loop() {
  // turn the LED on (HIGH is the voltage level)
  digitalWrite(LED_BUILTIN, HIGH); 

  delay(1000);      // wait for a second

  // turn the LED off by making the voltage LOW
  digitalWrite(LED_BUILTIN, LOW);  
  
  delay(1000);      // wait for a second
}
\end{lstlisting}


There are a few common aspects present in the code of nearly every Arduino project, no matter how simple or complicated. \\


\lstinline[]!pinMode(<pin number>, <mode>)! sets a digital pin on the Arduino to be either an \lstinline[]!INPUT! or and \lstinline[]!OUTPUT!. \\


\lstinline[]!digitalWrite()! is used to set digital pins \lstinline[]!HIGH! and \lstinline[]!LOW!. High can be thought of as turning on something and low is turning off something just like with the LED in the above example. 

Sometimes this will change, it depends on whether the component is active high (normally low, make it high to turn it on) or active low (normally high, make it low to turn it on).

\end{document}