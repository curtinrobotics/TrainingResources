\documentclass[../TinyBot.tex]{subfiles}
\begin{document}
    
\section{Motor} \label{sec:motor}
gearbox, motor

To follow this guide it is not necessary to have an understanding of how motors work, though it may be interesting for you to learn. \href{https://www.explainthatstuff.com/electricmotors.html}{This} link has a good indepth explanation.


Motors turn in proportion to the amount of current put through them. More current means a faster motor. 

When a motor stalls, it stops rotating. This happens when there is more force acting on the motor shaft than the motor can overcome. The stall torque of a motor is the maximum current drawn when a motor stalls, in other words, applying its maximum torque. \\


Similarly, free current is the current drawn when the motor is rotating freely, under no load. 

\bigskip


Each motor has a certain amount of torque it can provide. Gearboxes can be attached to a motor to increase the amount of torque provided, and change the rotations per minute (RPM) of the motor. 


% $TODO - Someone who knows about gearboxes, please complete this section. 

\end{document}