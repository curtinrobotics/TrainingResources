\documentclass[../TinyBot.tex]{subfiles}
\begin{document}
    
\section{Motor Controller} \label{sec:motorcontroller}

The motors used in this guide, the N20 motors, have a stall current of 1.6A (see section \ref{sec:motor} for what stall current means). The digital pins on an Arduino Uno supply at most 40mA. This is not enough to power the motors.\\ 

To get around this, the Arduino instead interfaces with a \textbf{motor controller}. Motor controllers have a separate power supply that can supply enough current to drive the motor. Motor controllers also have digital inputs that allow control of the motor. \\


An added benefit of using a motor controller is that it is possible to control the direction and the speed of the motor. 

\bigskip

The phrase motor controller is often used as a generic term for any device, circuit, or IC which controls a motor. However, motor controllers are a circuit that consists of a motor driver and some digital harness that acts as an interface to the driver. Motor controllers can be dropped into a circuit and easily controlled, allowing feedback from the motor and more control than a simple driver provides. \\


% An example of a motor driver versus a motor controller is the \href{https://www.altronics.com.au/p/z2900-l293d-motor-drive-ic/}{L293D} motor driver versus the \href{https://www.altronics.com.au/p/z6442-l298n-dual-h-bridge-motor-module-for-arduino/}{L298N}. 


\subsection{Motor Driver}
A basic motor-\textbf{driver} is a H-bridge. The simplest H-bridge is shown in the below schematics, as well as an explanation of how using a H-bridge allows control over the motors direction. 

\begin{center}
    \begin{circuitikz}
    \draw (0,2) -- (0,0) to
        (0,0) to[nos, l^=$1$] (2,0) to
        (2,0) to [nos, l^=$2$] (4,0) to
        (4,0) -- (4,2) to
        (4,2) to[nos, l^=$3$] (2,2) to
        (2,2) to[nos, l^=$4$] (0,2);
    \draw (2,1) node[elmech](motor){M};
    \draw (motor.north) -- (2,2);
    \draw (motor.south) -- (2,0);
    \draw (0,1) -- (-1,1) node[vee]{};
    \draw (4,1) -- (5,1) node[ground]{};
    % \draw (2,0) to[sV, color=white, name=M] (2,2);
    % \mymotor{M}{90};
    % \draw[rotate=2] (0,2) \mymotor{M}{90} (2,2);
    \end{circuitikz}
    % \captionof{figure}{}
\end{center}

When switches $1$ and $3$ are closed, the current will flow through the motor making it turn anticlockwise.
% in the motor will flow in one direction. 

\begin{center}
    \begin{circuitikz}
        \draw (0,2) -- (0,0) to
            % (0,0) to[o-o, l_=$1$] (2,0) to
            (0,0) -- node[above, yshift=1.5mm]{1} (2,0) to
            (2,0) to [nos, l^=$2$] (4,0) to
            (4,0) -- (4,2) to
            % (4,2) to[l_=$3$] (2,2) to
            (4,2) -- node[below, yshift=-1.5mm]{3} (2,2) to
            (2,2) to[nos, l^=$4$] (0,2);
        \draw (motor.north) -- (2,2);
        \draw (motor.south) -- (2,0);
        \draw[color=red!100, thick] (0,1) -- (-1,1) node[vee]{};
        \draw[color=red!100, thick] (4,1) -- (5,1) node[ground]{};
     
        \begin{scope}[>=latex]
            \draw[->, color=red!100, thick] (-1,1) -- (0,1);
            \draw[->, color=red!100, thick] (0,1) -- (0,0);
            \draw[->, color=red!100, thick] (0,0) -- (2,0);
            \draw[->, color=red!100, thick] (2,0) -- (2,2);
            \draw[->, color=red!100, thick] (2,2) -- (4,2); 
            \draw[->, color=red!100, thick] (4,2) -- (4,1);
        \end{scope}
        \draw (2,1) node[elmech](motor){M};
        \centerarc[->](2,1)(-45:45:0.7);
        \centerarc[->](2,1)(135:225:0.7);
    \end{circuitikz}
\end{center}

In the same vein, closing switches 2 and 4 will cause the motor to turn clockwise. 
\begin{center}
    \begin{circuitikz}
        \draw (0,2) -- (0,0) to
            (0,0) to[nos, l^=$1$] (2,0) to
            % (2,0) to [nos, l_=$2$] (4,0) to
            (2,0) -- node[above, yshift=1.5mm]{2}(4,0) to 
            (4,0) -- (4,2) to
            (4,2) to[nos, l^=$3$] (2,2) to
            % (2,2) to[nos, l_=$4$] (0,2);
            (2,2) -- node[below, yshift=-1.5mm]{4} (0,2);
        \draw (motor.north) -- (2,2);
        \draw (motor.south) -- (2,0);
        \draw[color=red!100, thick] (0,1) -- (-1,1) node[vee]{};
        \draw[color=red!100, thick] (4,1) -- (5,1) node[ground]{};

        \begin{scope}[>=latex]
            \draw[->, color=red!100, thick] (-1,1) -- (0,1);
            \draw[->, color=red!100, thick] (0,1) -- (0,2);
            \draw[->, color=red!100, thick] (0,2) -- (2,2);
            \draw[->, color=red!100, thick] (2,2) -- (2,0);
            \draw[->, color=red!100, thick] (2,0) -- (4,0); 
            \draw[->, color=red!100, thick] (4,0) -- (4,1);
        \end{scope}
        \draw (2,1) node[elmech](motor){M};

        \centerarc[->](2,1)(45:-45:0.7);
        \centerarc[->](2,1)(225:135:0.7);

    \end{circuitikz}
\end{center}

If pins 4 \& 3 or pins 1 \& 2 are closed at the same time, a short circuit will be formed and the H-bridge will break. \\

\begin{minipage}{0.5\textwidth}\vspace{0pt}
    \begin{center}
        \begin{circuitikz}
            \draw (0,2) -- (0,0) to
                (0,0) to[nos, l^=$1$] (2,0) to
                (2,0) to [nos, l^=$2$] (4,0) to
                % (2,0) -- node[below, yshift=-1.5mm]{2}(4,0) to 
                (4,0) -- (4,2) to
                % (4,2) to[nos, l_=$3$] (2,2) to
                % (2,2) to[nos, l_=$4$] (0,2);
                (4,2) -- node[below, yshift=-1.5mm]{3} (2,2) to
                (2,2) -- node[below, yshift=-1.5mm]{4} (0,2);
            \draw (motor.north) -- (2,2);
            \draw (motor.south) -- (2,0);
            \draw[color=red!100, thick] (0,1) -- (-1,1) node[vee]{};
            \draw[color=red!100, thick] (4,1) -- (5,1) node[ground]{};
            
            \begin{scope}[>=latex]
                \draw[->, color=red!100, thick] (-1,1) -- (0,1);
                \draw[->, color=red!100, thick] (0,1) -- (0,2);
                \draw[->, color=red!100, thick] (0,2) -- (2.1,2);
                \draw[->, color=red!100, thick] (2,2) -- (4.01,2);
                % \draw[->, color=red!100, thick] (2,0) -- (4,0); 
                \draw[->, color=red!100, thick] (4,2) -- (4,1);
            \end{scope}
            \draw (2,1) node[elmech](motor){M};
    
        \end{circuitikz}
    \end{center}

\end{minipage}
\begin{minipage}{0.5\textwidth}\vspace{0pt}
    \begin{center}
        \begin{circuitikz}
            \draw (0,2) -- (0,0) to
                % (0,0) to[nos, l_=$1$] (2,0) to
                % (2,0) to [nos, l_=$2$] (4,0) to
                (0,0) -- node[above, yshift=1.5mm]{1}(2,0) to 
                (2,0) -- node[above, yshift=1.5mm]{2}(4,0) to 
                (4,0) -- (4,2) to
                (4,2) to[nos, l^=$3$] (2,2) to
                (2,2) to[nos, l^=$4$] (0,2);
                % (2,2) -- node[above, yshift=1.5mm]{4} (0,2);
            \draw (motor.north) -- (2,2);
            \draw (motor.south) -- (2,0);
            \draw[color=red!100, thick] (0,1) -- (-1,1) node[vee]{};
            \draw[color=red!100, thick] (4,1) -- (5,1) node[ground]{};    
    
            \begin{scope}[>=latex]
                \draw[->, color=red!100, thick] (-1,1) -- (0,1);
                \draw[->, color=red!100, thick] (0,1) -- (0,0);
                \draw[->, color=red!100, thick] (0,0) -- (2.1,0);
                % \draw[->, color=red!100, thick] (2,2) -- (2,0);
                \draw[->, color=red!100, thick] (2,0) -- (4,0); 
                \draw[->, color=red!100, thick] (4,0) -- (4,1);
            \end{scope}
            \draw (2,1) node[elmech](motor){M};
    
        \end{circuitikz}
    \end{center}
\end{minipage}

\bigskip \bigskip

Breaking a H-bridge is fairly common, especially the cheaper low power ones. Some higher end H-bridges are designed to prevent the H-bridge from shorting if the wrong pins are closed. 
Most motor-controllers will have this protection built-in, though most motor-drivers do not.


While working on this guide, don't worry if your H-bridge stops working suddenly, it is quite common to short them out. 

\bigskip


% In the case of the H-bridge, the voltage supplied (from the left on the diagrams displayed above) is a high enough voltage, with a high enough current, to drive the motor. The switches (1, 2, 3, 4) are, in their simplest form, a button that is closed upon a signal from the microcontroller. Switches require a very small amount of power to close, and so can be controlled directly from the motor controller. 



\subsection{Motor Controller}
% https://www.robotshop.com/en/roboclaw-2x30a-6-34vdc-regenerative-motor-controller.html


\begin{wrapfigure}[10]{l}{0.4\textwidth}
    \includegraphics[width=0.4\textwidth]{roboclaw.jpg}
    \caption{\href{https://www.robotshop.com/media/files/content/b/bat/pdf/roboclaw_datasheet_2x30a-2.pdf}{RoboClaw Dual Channel DC Motor Controller}}
    \label{fig:roboclaw}
\end{wrapfigure}

A motor controller has a lot more features than a motor driver. See, for example, the RoboClaw (see Figure \ref{fig:roboclaw}) which has in built features such as PID tuning, data logging, diagnostic LEDs, and serial control. \\


The in-built control modes, as well as being capable of serial communication, is present only in motor controllers. Motor drivers are far simpler in comparision. \\

\end{document}